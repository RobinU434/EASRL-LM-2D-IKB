\chapter*{Abstract}
Reinforcement Learning (RL) is a powerful paradigm for training agents to make sequential decisions in dynamic environments. This thesis explores innovative approaches to address the challenge of expanding the action space of RL agents. This research leverages latent models, specifically Variational Autoencoders (VAEs) and artificial neural networks trained in a supervised fashion, with the goal to transform the action space, enabling RL agents to handle a bigger action space.

The investigation centers on the application of these latent models to a 2D Inverse Kinematics problem in a robotic arm with numerous joints. Through a careful analysis of experimental results, the thesis assesses the impact of action space enlargement on RL agent performance and scalability.

This thesis discusses tey findings and contributions to the advantages and limitations of latent model integration and the challenges of high-dimensional action spaces for optimizing RL agents' decision-making capabilities.

% As RL continues to evolve and find application in various domains, this thesis sheds light on novel techniques for improving agent capabilities, fostering adaptability, and advancing the field's understanding of action space expansion in complex scenarios.

By bridging the gap between latent models and RL, this research provides a foundation for future investigations into enhancing the capabilities of autonomous agents.

\chapter*{Zusammenfassung}

Selbstbestimmtes Lernen (Reinforcement Learning, RL) ist ein leistungsstarkes Paradigma zur Schulung von Agenten, um sequenzielle Entscheidungen in dynamischen Umgebungen zu treffen. Diese Arbeit erforscht Ansätze zur Bewältigung der Herausforderung, den Handlungsraum von RL-Agenten zu erweitern. Diese Thesis nutzt latente Modelle, insbesondere Variational Autoencoders (VAEs) und künstliche neuronale Netze mit dem Ziel, den Handlungsraum zu transformieren und zu vergrößern, um RL-Agenten in die Lage zu bringen, einen größeren Handlungsraum zu bewältigen.

Die Untersuchung konzentriert sich auf die Anwendung dieser latenten Modelle auf ein 2D-Inverse-Kinematikproblem in einem Roboterarm mit zahlreichen Gelenken. Durch eine Analyse der experimentellen Ergebnisse bewertet die Arbeit die Auswirkungen der Vergrößerung des Handlungsraums auf die Leistung und die Skalierbarkeit von RL-Agenten.

Diese Arbeit diskutiert hierbei die Hauptergebnisse und Beiträge zu den Vor- und Nachteile der Integration latenter Modelle und die Herausforderungen von hochdimensionalen Handlungsräumen zur Optimierung der Entscheidungsfähigkeiten von RL-Agenten.

% Während sich RL weiterhin in verschiedenen Bereichen weiterentwickelt und Anwendung findet, wirft diese Arbeit Licht auf neue Techniken zur Verbesserung der Agentenfähigkeiten, Förderung der Anpassungsfähigkeit und Vertiefung des Verständnisses für die Erweiterung des Handlungsraums in komplexen Szenarien.

Indem sie die Kluft zwischen latenten Modellen und RL überbrückt, legt diese Forschung die Grundlage für zukünftige Untersuchungen zur Verbesserung der Fähigkeiten autonomer Agenten.