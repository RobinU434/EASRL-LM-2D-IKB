\chapter{Introduction}\label{chap:introduction}



% This thesis explores the integration of latent models, including pre-trained Variational Autoencoder decoders and supervised models, into Reinforcement Learning frameworks. The objective is to expand the action space capabilities of RL agents. The research is evaluated using a benchmark environment: solving a 2D Inverse Kinematics problem on a robot arm with a variable number of joints $N$. The study investigates how the incorporation of latent models enhances the RL agent's performance in handling complex action spaces.

In the ever-evolving landscape of artificial intelligence and machine learning, Reinforcement Learning (RL) stands as a powerful paradigm for enabling autonomous agents to learn and adapt to their environments. RL has demonstrated remarkable success in a variety of applications, from game-playing agents to robotics control systems. However, one of the enduring challenges in RL lies in the limitation of the action space—an agent's ability to act upon its environment. The dimensionality and complexity of the action space profoundly impact an agent's capacity to solve increasingly intricate tasks.

This thesis explores a novel approach to address the action space limitations in RL by leveraging latent models. Specifically, it investigates the integration of pre-trained latent models, including Variational Autoencoder (VAE) decoders and supervised models, into RL frameworks. The primary aim is to empower RL agents to handle significantly more extensive and complex action spaces effectively. To evaluate the effectiveness of this approach, we developed a new scalable benchmark environment: the 2D Inverse Kinematics problem.

\textbf{The Challenge of Action Space in Reinforcement Learning}

The action space of an RL agent comprises the set of actions it can take to interact with its environment. In many real-world scenarios, especially those involving robotic systems and control tasks, the action space can be high-dimensional and continuous. Traditional RL algorithms often struggle to operate effectively in such settings due to the curse of dimensionality, making the exploration of all possible actions computationally infeasible.

This challenge has motivated researchers to seek innovative solutions that expand the boundaries of RL action spaces. The integration of latent models offers a promising avenue. These models can encode complex actions into lower-dimensional representations, enabling RL agents to navigate large action spaces more efficiently and effectively.

\textbf{The Role of Latent Models in Expanding Action Spaces}

Latent models, such as VAEs and supervised models, have demonstrated their prowess in capturing essential patterns and representations within data. By pre-training these models on relevant tasks, we can harness their latent spaces to transform high-dimensional action spaces into more compact and manageable forms. This, in turn, equips RL agents with the ability to explore and exploit action spaces that were previously deemed insurmountable.

\textbf{Benchmarking Progress: The 2D Inverse Kinematics Problem}

To assess the efficacy of integrating latent models into RL, we turn to the 2D Inverse Kinematics problem. This problem simulates the control of a multi-jointed robotic arm, where each joint represents a dimension in the action space. The challenge lies in determining the joint angles required to position the end effector at a specified target location. By scaling the complexity of this benchmark through varying the number of joints $N$, we can rigorously evaluate the impact of latent models on action space expansion.

\textbf{Structure of the Thesis}

This thesis is structured as follows: in \chapref{chap:background}, we provide an overview of the background for this research. \chapref{chap:relatedwork} delves into a comprehensive literature review, examining related work in RL, latent models, and their integration. \chapref{chap:Methodology} outlines the methodology used, including details on the latent models employed and their integration into RL. \chapref{chap:experiments} presents the experimental results and analysis from the 2D Inverse Kinematics benchmark. \chapref{chap:discussion} discusses experimental results and finally, \chapref{chap:conclusion} offers conclusions and outlines potential future directions for research in this domain.

The integration of latent models with RL to expand action space capabilities holds great promise for advancing the capabilities of autonomous agents. By addressing one of the core challenges in RL, this research aims to contribute to the broader field of artificial intelligence and robotics, opening doors to new possibilities in complex task execution and problem-solving.




% This is a template for an undergraduate or master's thesis.
% The first sections are concerned with the template itself. If this is your first
% thesis, consider reading \secref{sec:advice}.
% 
% The structure of this thesis is only an example.
% Discuss with your adviser what structure fits best for your thesis.
% 
% \section{Template Structure}
% \begin{itemize}
%     \item To compile the document either run the makefile or run your compiler on the file `thesis\_main.tex'. The included makefile requires latexmk which automatically runs bibtex and recompiles your thesis as often as needed. Also it automatically places all output files (aux, bbl, ...) in the folder `out'. As the pdf also goes in there, the makefile copies the pdf file to the parent folder. There is also a makefile in the chapters folder, to ensure you can also compile from this directory.
% 
%     \item The file `setup.tex' includes the packages and defines commands. For more details see \secref{sec:setup}.
% 
%     \item Each chapter goes into a separate document, the files can be found in the folder chapters.
% 
%     \item The bib folder contains the .bib files, I'd suggest to create multiple bib files for different topics. If you add some or rename the existing ones, don't forget to also change this in thesis\_main.tex. You can then cite as usual~\cite{kingma2014adam, bromley1993siamesesignature,muja2009flann}.
% 
%     \item The template is written in a way that eases the switch from scrbook to book class. So if you're not a fan of KOMA you can just replace the documentclass in the main file. The only thing that needs to be changed in setup.tex is the caption styling, see the comments there.
% \end{itemize}
% 
% 
% \section{setup.tex}\label{sec:setup}
% Edit setup.tex according to your needs. The file contains two sections, one for package includes, and one for defining commands. At the end of the includes and commands there is a section that can safely be removed if you don't need algorithms or tikz. Also don't forget to adapt the pdf hypersetup!!\\
% setup.tex defines:
% \begin{itemize}
%     \item some new commands for remembering to do stuff:
%     \begin{itemize}
%         \item \verb|\todo{Do this!}|: \todo{Do this!}
%         \item \verb|\extend{Write more when new results are out!}|:\\ \extend{Write more when new results are out!}
%         \item \verb|\draft{Hacky text!}|: \draft{Hacky text!}
%     \end{itemize}
% 
%     \item some commands for referencing, `in \verb|\chapref{chap:introduction}|' produces 'in \chapref{chap:introduction}'
%     \begin{itemize}
%         \item \verb|\chapref{}|
%         \item \verb|\secref{sec:XY}|
%         \item \verb|\eqref{}|
%         \item \verb|\figref{}|
%         \item \verb|\tabref{}|
%     \end{itemize}
% 
%     \item the colors of the Uni's corporate design, accessible with\\ \verb|{\color{UniX} Colored Text}|
%     \begin{itemize}
%         \item {\color{UniBlue}UniBlue}
%         \item {\color{UniRed}UniRed}
%         \item {\color{UniGrey}UniGrey}
%     \end{itemize}
% 
%     \item a command for naming matrices \verb|\mat{G}|, $\mat{G}$, and naming vectors \verb|\vec{a}|, $\vec{a}$. This overwrites the default behavior of having an arrow over vectors, sticking to the naming conventions  normal font for scalars, bold-lowercase for vectors, and bold-uppercase for matrices.
% 
%     \item named equations:
%         \begin{verbatim}
% \begin{align}
%     d(a,b) &= d(b,a)\\ \eqname{symmetry}
% \end{align}
%         \end{verbatim}
%         \begin{align}
%             d(a,b) &= d(b,a)\\ \eqname{symmetry}
%         \end{align}
% \end{itemize}
% 
% \section{Advice}\label{sec:advice}
% This section gives some advice how to write a thesis ranging from writing style to formatting. To be sure, ask your advisor about his/her preferences.\\
% For a more complete list we recommend to read Donald Knuth's paper on mathematical writing. (At least the first paragraph). \url{http://jmlr.csail.mit.edu/reviewing-papers/knuth_mathematical_writing.pdf}
% 
%     \begin{itemize}
% 
%         \item If you use formulae pay close attention to be consistent throughout the thesis!
% 
%         \item In a thesis you don't write `In [24] the data is..'. You have more space than in a paper, so write `AuthorXY et al. prepare the data... [24]'. Also pay attention to the placement: The citation is at the end of the sentence before the full stop with a no-break space. \verb|... last word~\cite{XY}.|
% 
%         \item Pay attention to comma usage, there is a big difference between English and German. `...the fact that bla...' etc.
% 
%         \item Do not write `don't ', `can't' etc. Write `do not', `can not'.
% 
%         \item If an equation is at the end of a sentence, add a full stop. If it's not the end, add a comma: {$a= b + c$~~~~(1),}
% 
%         \item Avoid footnotes if possible.
% 
%         \item Use \verb|``''| for citing, not \verb|""|.
% 
%         \item It's important to look for spelling mistakes in your thesis. There are also tools like aspell that can help you find such mistakes.
%         This is never an excuse not to properly read your thesis again, but it can help.
%         You can find an introduction under \url{https://git.fachschaft.tf/fachschaft/aspell}.
% 
%         \item If have things like a graph or any other drawings consider using tikz, if you need function graphs or diagrams consider using pgfplots.
%         This has the advantage that the style will be more consistent (same font, formatting options etc.) than when you use some external program.
% 
%         \item Discuss with your advisor whether to use passive voice or not. In most computer science papers passive voice is avoided. It's harder to read, more likely to produce errors, and most of the times less precise. Of course there are situations where the passive voice fits but in scientific papers they are rare. Compare the sentence: `We created the wheel to solve this.' to `The wheel was created to solve this', you don't know who did it, making it harder to understand what is your contribution and what is not.
%         
%         \item In tables avoid vertical lines, keep them clean and neat. See \ref{tab:accuracy} for an example. More details can be found in the `Small Guide to Making Nice Tables' \url{https://www.inf.ethz.ch/personal/markusp/teaching/guides/guide-tables.pdf}
% 
%     \end{itemize}
% 