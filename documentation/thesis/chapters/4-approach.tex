\chapter{Approach}\label{chap:approach}
The approach starts with the problem definition and continues with what you have done. Try to give an intuition first and describe everything with words and then be more formal like `Let $g$ be ...'.

\section{Problem Definition}
% Start with a very short motivation why this is important. Then, as stated above, describe the problem with words before getting formal.
The problem we concentrate is to solve inverse kinematics for a varying number of joints. This setting was chosen because it satisfies two properties. It has an scaleable action space and is easy to extendable to make it more complex so standard solver like Cyclic-Coordenate-Descent would fail \todo{show that CCD would fail}. \\
But before we started to make the environment more complex we started with solving inverse kinematics without any constraints.

\subsection{the environment}

\section{Variational Autoencoder}

In this section we will discuss the creation of different datasets to solve the inverse kinematics problem.
\todo{very similar actions with CCD}
\todo{Show: there are multiple actions for a single target position, https://www.alanzucconi.com/2018/05/02/ik-2d-1/ for two joints}

\subsection{Pure Actions}

\subsection{Conditioning on the States}
\todo{pseudo code}
 
\subsection{Fitting Random Noise}

 
\section{Supervised learning}

In this section we will discuss the approach to solve the inverse kinematics problem with supervised learning. This approach was only chosen to show that inverse kinematics is solveable with a neural network


\subsubsection{distance function}
\begin{equation}
    \mathcal{L}(x, \hat{x}) = \frac{1}{2} \cdot \sum_{i\in \{0 ,1\}} (x_i - fk(\hat{x})_i)
\end{equation}

$fk$ is forward kinematics

\section{SAC}

\subsection{MLP}

\subsection{RNN}

\subsection{imitation learning}

\subsection{hyperparamter tuning}
\todo{make a description about the tuning of action magnitude and target entropy}

\subsection{strategic vs one shot}

\todo{look at the code and }

\subsection{latent actor}

\subsection{Problems with exploration}

\subsection{actions with constraints}